\documentclass{acmsiggraph}
\usepackage[scaled=.92]{helvet}
\usepackage{times}

\usepackage{rotating}
\newcommand{\rotxc}[1]{\begin{sideways}#1\end{sideways}}
\newcommand{\invert}[1]{\rotxc{\rotxc{#1}}}
\usepackage{graphicx}

%% use this for zero \parindent and non-zero \parskip, intelligently.

\usepackage{parskip}
\usepackage{footnote}

\renewcommand{\thefootnote}{\fnsymbol{footnote}}

\usepackage[labelfont=bf,textfont=it]{caption}
\usepackage{amsmath, amsthm, amssymb}

\newtheorem{thm}{Theorem}[section]
\newtheorem{cor}[thm]{Corollary}
\newtheorem{lem}[thm]{Lemma}

\theoremstyle{remark}
\newtheorem{rem}[thm]{Remark}

\theoremstyle{definition}
\newtheorem{defn}[thm]{Definition}

%Hacky pseudo-code:
\newcommand{\cF}[1]{{\tt #1}}
\newcommand{\cTAB}{\phantom{----}}
\newcommand{\cIF}{{\bf if}}
\newcommand{\cRET}{{\bf return}}
\newcommand{\cWHILE}{{\bf while}}
\newcommand{\cFOR}{{\bf for}}

\newcommand{\defbreak}{\vspace{1em}}
\title{The {\em fluint8} Software Integer Library}

\author{Jim McCann\thanks{e-mail: ix@tchow.com}\\TCHOW llc \and Your Name Here}

%\teaser{
%\centering
%\Huge
%A ? B : C
%\caption{\label{fig-ternary}
%The ternary traditional operator (``tradnary operator'') evaluates to either B or C depending on the truth of operand A.
%}}

%\keywords{anger, invective, morality}
\begin{document}

\maketitle

\begin{abstract}
We present {\em fluint8}, an library for performing integer math, including basic arithmetic and bitwise logical operations, using only basic floating point operations.
\end{abstract}

\begin{CRcatlist}
\CRcat{\_.\_.\_}{Blank}{Blankings}{\small Blank}
\end{CRcatlist}

\section{Introduction}
There are a surfeit of libraries that exist to perform floating point operations on processors that only support integer math.
This is unsurprising, as many such processors exist -- from ancient 286's to modern embedded microcontrollers.
These libraries use many integer instructions to emulate the action of a floating point unit, providing correct and useful (if slow) results.

We present a small header-only library to emulate integer operations -- specifically 8-bit unsigned integer operations -- using standard IEEE 754 single-precision (binary32) floating point math.
Our presented operations have been designed to be succinct but also pleasantly puzzling.

As far as we are aware, no processor exists for which this library would be required.
However, perhaps you should consider that a challenge.

\section{Floating Point}
TODO: image of floating point number either here or in the teaser

An IEEE 754 single-precision floating point number (binary32 format) is stored as a sign bit, a 8-bit exponent, and a 23-bit mantissa.
Except for special cases, the number represented by a floating point number with sign $S$, exponent $E$, and mantissa $M$ is
\footnote{or at least this is what wikipedia says, so I'm going with that, and it seems to work out.}:
\begin{displaymath}
-1^{S}\cdot (1.M)_2 \cdot 2^{E-127}
\end{displaymath}

Particularly, notice that the leading ``1'' in the fraction is implicit in the representation (it is implied by the exponent).

This means that the range of integers that can be represented (without loss of precision) is
\begin{displaymath}
[-2^{24},2^{24}] = [-16777216, 16777216]
\end{displaymath}
which, conveniently, is far more than the $[0,255]$ range needed for storing 8-bit unsigned integers.

When floating point operations result in numbers that cannot be accurately represented, the results are rounded according to the current rounding mode.
The default rounding mode assumed in this paper is {\em roundTiesToEven}.
I would say that it does what you expect, but floating point numbers seldom manage that feat.
Regardless, this rounding mode means that whenever a value is exactly half way between two representable numbers, the number with a least-significant-bit of 0 is picked.

Rounding and precision loss leads to this fun fact:
\begin{center} \tt
16777216.0f + 1.0f - 1.0f == 16777215.0f\\
16777216.0f - 1.0f + 1.0f == 16777216.0f
\end{center}
(Hot take: floating-point operations are non-commutative.)

\section{Storage Format}
Our library represents unsigned 8-bit integers as their equivalent floating point values.
In other words, the value {\tt uint\_t(127)} is represented as {\tt 127.0f}.
This straightforward equivalence makes it very easy to write basic mathematical functions.

\section{Arithmetic Functions}
Our library implements {\tt +}, {\tt -}, {\tt *}, {\tt /}, and {\tt -} by treating floating point numbers as real numbers; an approach that often works:

{\tt
float fu8\_add(float a, float b) \{ \\
$\phantom{XX}$return fmodf(a+b,256.0f); \\
\} \\
float fu8\_sub(float a, float b) \{ \\
$\phantom{XX}$return fmodf(a+256.0f-b,256.0f); \\
\} \\
float fu8\_mul(float a, float b) \{ \\
$\phantom{XX}$return fmodf(a*b,256.0f); \\
\} \\
float fu8\_div(float a, float b) \{ \\
$\phantom{XX}$return floorf(a/b); \\
\} \\
float fu8\_neg(float a) \{ \\
$\phantom{XX}$return fmodf(256.0f-a,256.0f); \\
\}
}

Notice that the problem of figuring out how to, e.g., implement branching and conditional tests are left to the standard library functions.

\section{Conversion Functions}
In order to support, e.g., reading data from files, our library includes functions that convert between floating point numbers and bit-patterns of their equivalent 8-bit unsigned representation.
This is where the code starts to get fun.

{\tt
void fu8\_to\_bits(float a, void *out) \{ \\
$\phantom{XX}$a += 8388608.0f; \\
$\phantom{XX}$memcpy(out, \&a, size\_t(1.0f)); \\
\}
}

What this code does is adds a large enough number to {\tt a} that its mantissa's least-significant bit now represents 1.
Essentially, the code is shoving the integer information stored in {\tt a} to the least-significant-byte of the representation, and then copying\footnote{
The astute reader will notice that we've taken care to avoid using an integer constant as a parameter to {\tt memcpy}.
Presumably on processors without integer support {\tt size\_t} must be a floating-point type.
} it out to the destination.

The same trick works when setting a floating point number from an integer bit pattern:

{\tt
float fu8\_from\_bits(void const *from) \{ \\
$\phantom{XX}$float a = 8388608.0f; \\
$\phantom{XX}$memcpy(\&a, from, size\_t(1.0f)); \\
$\phantom{XX}$return a - 8388608.0f; \\
\}
}


\section{Bitwise Operations}

Things really get interesting when we begin to look at bitwise operations, which aren't standard operations on floating point numbers\footnote{Though they seem well-defined; maybe a language-designer oversight?}.

Let's begin with bitwise negation ({\tt ~}).
This one is relatively easy to explain -- an unsigned 8-bit integer plus its bitwise complement is always 255, which makes negation as easy as subtraction:

{\tt
float fu8\_not(float a) \{ \\
$\phantom{XX}$return 255.0f - a; \\
\}
}

Things get a bit more interesting when computing bitwise and ({\tt \&}):

{\tt
float fu8\_and(float a, float b) \{ \\
$\phantom{XX}$float ax, bx, x = 0.0f; \\
$\phantom{XX}$for (float c = 2147483648.0f;\\
$\phantom{XXXXXX}$ c != 8388608.0f; c *= 0.5f) \{ \\
$\phantom{XXXX}$a -= ax = (a + 1.0f + c-c)/ 2.0f; \\
$\phantom{XXXX}$b -= bx = (b + 1.0f + c-c)/ 2.0f; \\
$\phantom{XXXX}$x = 0.5f * x + ax * bx; \\
$\phantom{XX}$\} \\
$\phantom{XX}$return x; \\
\}
}

Note that though this is presented as a loop, the loop has constant bounds and could be unrolled by a compiler into eight repetitions of the same code.

So, what sort of trickery is going on here?
Obviously this code is treating {\tt a} and {\tt b} bit-by-bit, but how?
The key turns out to be this little waving guy; we'll call it Milt:

\begin{center}
\huge
{\tt c-c)/}
\end{center}

Though it looks like Milt is just hanging out, minding its own business, and not changing the value of the expression,
Milt is in fact doing something surprisingly nonlinear.

TODO: plots of the effect of Milt.

So when Milt's eyes are {\tt 2147483648.0f}, it is extracting twice the value of the MSB of {\tt a},
which in turn is stored in {\tt ax} and subtracted from {\tt a}.
In this way, the code peels off each successive most-significant bit from {\tt a} and {\tt b} and accumulates their product to the final result.

This leaves only the mystery of why {\tt x} is being divided by two each loop iteration.
But this isn't a mystery at all.
Consider computing {\tt 128 \& 255}.
Notice that on the first iteration, the product {\tt 128.0f * 128.0f} would be added to {\tt x};
the multiplication by a factor of {\tt 0.5f} on each subsequent iteration simply -- in aggregate -- bring it to the correct result of {\tt 128.0f}.

{ \tt
\begin{tabular}{r|r|r|r|r}
\multicolumn{1}{c|}{a} &
\multicolumn{1}{c|}{b} &
\multicolumn{1}{c|}{ax} &
\multicolumn{1}{c|}{bx} &
\multicolumn{1}{c}{x} \\\hline
177.0f & 233.0f & 128.0f & 128.0f & 16384.0f \\
\end{tabular}
}

TODO: fill in the rest of the table with an example run.

\section{Future Work}

Design a processor for which this library is relevant.

\end{document}
